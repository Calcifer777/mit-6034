\section{Symbolic constraints and propagation}

\paragraph{Principle of convergent intelligence} The world
manifest constraints and regularities. If a computer is to
exhibit intelligence, it must exploit those constraints and
regularities, no matter of what the computer happens to be made.

\paragraph{Describe-to-explain principle} The act of detailed
description may turn probabilistic regularities into entirely
deterministic constrains.

\paragraph{Marr's methodological principles}
\begin{enumerate}
  \item Identify the problem
  \item Select or develop an appropriate representation
  \item Expose constraints or regularities
  \item Create particular procedures 
  \item Verify via experiments
\end{enumerate}

\paragraph{Contraction net} is a representation that is a \textit{frame
system}, in which:
\begin{itemize}
  \item Lexically and structurally, certain frame classes
    identify a finite list of application-specific
    interpretations
  \item Procedurally, demon procerures enforce compatibility
    constraints among connected frames
\end{itemize}

\subsection{Applications}

\subsubsection{3D object recognizion in 2D images}

\paragraph{Labeled drawing} is a representation that is a \textit{frame
system}, in which:
\begin{itemize}
  \item Lexically, there are line frames and junctions frames.
    Lines may be convex, cocave, or boundary lines. Junctions may
    be \textit{L}, \textit{Fork}, \textit{T}, or \textit{Arrow}
    junctions
  \item Structurally, junctions frames are connected by line
    frames. Also, each junction frame contains a list of
    interpretation combinations for its connecting lines
  \item Semantically, line frames denotee physical edges.
    Junction frames denote physical vertexes
  \item Procedurally, demon procedures enforce the constraint
    that each junction lavel must be compatible with at least one
    of the junction labels at each of the neighboring junctions
\end{itemize}

\subsubsection{Time-interval relations and scheduling}

\paragraph{Interval net} is a representation that is a
\textit{contraction net}, in which:
\begin{itemize}
  \item Lexically and semantically there are interval frames
    denoting time intervals and lik frames denoting time
    relations: specifically: $\overrightarrow{before}$,
    $\overrightarrow{during}$, $\overrightarrow{overlaps}$,
    $\overrightarrow{meets}$, $\overrightarrow{starts}$,
    $\overrightarrow{finishes}$, $\overrightarrow{is equal to}$,
    and their mirrors
  \item Structurally, interval frames are connected by link
    frames
  \item Procedurally, demon procedures enforce the constraint
    that the interpretations allowed for a link frame between two
    intervals must be consistent with the interpretations allowed
    for the two link frames joining the two intervals to a third
    interval
\end{itemize}

\subsection{Map coloring - Lecture notes}

\begin{definition}
  A \textbf{variable} $v$ is something that can have an assigment
\end{definition}

\begin{definition}
  A \textbf{value} $x$ is something that can be an assigment
\end{definition}

\begin{definition}
  A \textbf{domain} $D$ is a bag of values
\end{definition}

\begin{definition}
  A \textbf{constraint} $c$ is a limit on variable values
\end{definition}

Procedure:
\begin{itemize}
  \item For each \textit{DFS} assignment
  \item For each variable $v_i$ considered.  A \textit{considered 
    variable} is some variable that may affect the assignment decision
  \item For each $x_i$ in $D_i$
  \item For each constraint $c(x_i, x_j)$ where $x_j \in D_j$
  \item If $\nexists x_j$ such that $c(x_i, x_j)$ is satisfied,
    then remove $x_i$ from $D_i$.
  \item If $D_i$ is empty, backtrack
\end{itemize}

Examples of \textit{considered variables}:
\begin{itemize}
  \item Nothing: leads to wrong outputs
  \item Assignment: constraints are too slack
  \item Neighbors' only assignment: can succeed but depends on
    assigment ordering. If most constrained assignments are
    considered first, then it is fast; vice versa, might not end
  \item Propagate checking through variables with reduced domain
    reduced to a single value
  \item Propagate checking through variables with reduced domains 
  \item Everything: too computationally intensive
\end{itemize}

