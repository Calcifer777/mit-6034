\documentclass{article}

\begin{document}

\title{Mit 6.043 - Artificial Intelligence}
\author{Marco Filippone}
\maketitle

\section{Reasoning: goal trees and rule-based expert systems}

A \textbf{semantic net} is a representation in which:
\begin{itemize}
  \item Lexically, there are nodes, links, and application-specific link labels
  \item Structurally, each link conects a tail node to a head node
  \item Semantically, the nodes and links denote application-specific entities
\end{itemize}
With constructors that:
\begin{itemize}
  \item Construct a node
  \item Construct a link, given a link label and two nodes to be connected
\end{itemize}
With readers that:
\begin{itemize}
  \item Produce a list of all links departing from a given node
  \item Produce a list of all links arriving at a given node
  \item Produce a tail node, given a link
  \item Produce a head node, given a link
  \item Produce a link label, given a link
\end{itemize}

A \textbf{Semantic tree} is a representation, that is a semantic net in which:
\begin{itemize}
  \item certain links are called *branches*. Each banch connects two nodes; the head node is called the *parent node* and the tail node is called the \textit{child node}
\item One node has no parent; it is called the root node. Other nodes have exactly one parent
\item Some nodes have no children, they are called \textit{leaf nodes}. When two nodes are connected to each other by a chain of two or more branches, one is said to be the \textit{ancestor}; the other if said to be the descendant
\end{itemize}

With constructors that:
\begin{itemize}
\item Connect a parent node to a child node with a branch links
\end{itemize}
With readers that:
\begin{itemize}
\item Produce a list of a given node's children
\item Produce a given node's parent
\end{itemize}

A \textbf{goal tree} is a semantic tree in which: nodes represent goals and branches indicate how you can achive goals by solving one or more subgoals. Each node's children corresponds to \textbf{immediate subgoals}; each node's parent corresponds to the \textbf{immediate supergoal}. The top node, the one with no parent, is the \textbf{root} goal.

Some goals are satisfied directly, without reference to any other subgoals. These goals are called \textbf{leaf goals}, and the corresponding nodes are called \textbf{leaf nodes}.

Because goal trees always involve \textit{And} nodes, or \textit{Or} nodes, or both. they are often called \textbf{And-Or trees}.

To determine whether a goal has been achieved, you need a testing procedure. The key procedure, \textit{REDUCE}, channels action into the \textit{REDUCE-AND} and the \textit{REDUCE-OR}.

Goal trees enable introspective question answering:
\begin{itemize}
  \item how: the immediate subgoal (downstream) 
  \item why the immediate supergoal (downstream) 
\end{itemize}

\subsection{Eliciting expert systems features}
\begin{itemize}
  \item Heuristic of specific situations: it is dangerous to limit inquiry to office interviews
  \item Heuristic of situation comparison: ask a domain expert for clarification whenever the domain expert's behavior varies in situations that look identical to the knowledge enginner.
  \item You should build a system and see when it cracks. Helps identifiying missing rules.
\end{itemize}






\end{document}
