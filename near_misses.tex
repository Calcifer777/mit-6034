\section{Near Misses and Felicity Conditions}


\paragraph{Near miss} A \textit{near miss} is a negative example
that, for a small number of reasons, is not an instance of the
class being taught.

\subsection{Heuristics list}
\begin{itemize}
  \item The \textbf{require-link} heuristic is used when an
    evolving model has a link in a place where a near miss does
    not. The model link is converted to a \textit{Must} form
  \item The \textbf{forbid-link} heuristic is used when a near miss
    has a link in a place where and evolving model does not. A
    \textit{Must-not} form is installed in te evolving model
  \item The \textbf{climb-tree} heuristic is used when an object in
    an evolving model corresponds to a different object in an
    example. \textit{Must-be-a} links are routed to the most
    specific common class in the classification tree above the
    model object and the example object
  \item The \textbf{enlarge-set} heuristic is used when an object
    in an evolving model corresponds to a different object in an
    example and the two objexts are not related to each other
    through a classification tree. \textit{Must-be-a} links are
    routed to a new class composed of the union of the objects'
    classes
  \item The \textbf{drop-link} heuristic is used when the objects
    that are different in an evolving model and in ana example form
    an exhaustive set. The drop-link heuristic is also used when
    an evolving model has a link that is not in the example. The
    link is dropped from the model
  \item The \textbf{close-interval} heuristic is used when a number
    or interval in an evolving model corresponds to a number in an
    example. If the model uses a number, the number is replaced by
    an interval spanning the model's number and the example's
    number. If the model uses an interval, the interval is enlarged
    to read the example's number
\end{itemize}

\subsection{Procedures}

\paragraph{SPECIALIZE}
\begin{itemize}
  \item Match the evolving model to the example to establish
    correspondences among parts
  \item Determine whether there is a single, most important
    difference between the evolving model and the near miss
  \begin{itemize}
    \item If there is a single, most important difference,
      \begin{itemize}
        \item If the evolving model has a link that is not in the
          near miss, use the require-link heuristic
        \item If the near miss has a link that is not in the model,
          use the forbid-link heuristic
      \end{itemize}
    \item Otherwise, ignore the example
  \end{itemize}
\end{itemize}

\paragraph{GENERALIZE}
\begin{itemize}
  \item Match the evolving model to the example to establish
    correspondences among parts
  \item For each difference, determine the difference type:
    \begin{itemize}
      \item If a link points to a class in the evolving model
        different from the class to which the link points in the
        example:
        \begin{itemize}
          \item If the classes are part of a class tree, use the
            climb-tree heuristic
          \item If the classes for an exhaustive set, use the
            drop-link heuristic
          \item Otherwise, use the enlarge-set heuristic
        \end{itemize}
      \item If a link is missing in the exmaple, use the drop-link
        heuristic
      \item If the difference is that different numbers, or an
        interval and a number out the interval, are involved, use
        the close-interval heuristic
      \item Otherwise, ignore the difference
    \end{itemize}
\end{itemize}

Notes:
\begin{itemize}
  \item Specialize does nothing if it cannot identify a most
    important difference. You need a procedure that ranks all
    differes by difference type and by link type
  \item Both SPECIALIZE and GENERALIZE involve matching. You need a
    matching procedure
\end{itemize}

\subsection{Learning}

To learn using procedure W:
\begin{itemize}
  \item Let the description of the first example, which must be an
    example, be the initial description
  \item For all subsection examples:
    \begin{itemize}
      \item If the example is a near miss, use SPECIALIZE
      \item If the example is an example, use GENERALIZE
    \end{itemize}
\end{itemize}

\paragraph{Wait-and-see principle} If there is doubt about 
what to do, do nothing.

Learning-facilitating teacher-student agreements are called
\textbf{felicity conditions}.

\paragraph{No-altering principle} When an object ir situation known
to be an example fails to match a general model, create a
special-case exception model.

\subsection{Identification}

\paragraph{Similarity net} A similarity net is a representation
that is a semantic net in which:
\begin{itemize}
  \item Nodes denotes models
  \item Links connect similar models
  \item Links are tied to difference descriptions
\end{itemize}

