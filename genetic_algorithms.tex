\section{Genetic Algorithms}

\paragraph{Chromosome} A chromosome is a representation in which:
\begin{itemize}
  \item There is a list of lements called genes
  \item The chromosome determines the overall fitness manifested by
    some mechanism that uses the chromosome's genes as a sort of
    blueprint
\end{itemize}
With constructors that:
\begin{itemize}
  \item Create a chromosome, given a list of elements; this
    constructor might be called the \textit{genesis constructor}
  \item Create a chromosome by crossing a pair of existing
    chromosomes
\end{itemize}
With writers that:
\begin{itemize}
  \item Mutate an existing chromosome by changing one of the genes
\end{itemize}
With readers that:
\begin{itemize}
  \item Produce a specified gene, given a chromosome
\end{itemize}

\subsection{Fitness}

The \textbf{standard method} for fitness computation is:
\begin{math}
  f_i = \frac{q_i}{\sum_{j}{q_j}}
\end{math}

The \textbf{rank method} for fitness computation is:
\begin{itemize}
  \item Sort the $n$ individuals by quality
  \item Let the probability of selecting the $i$th candidate, given
    that the first $i-1$ candidates have not been selected, be $p$,
    except for the final candidate, which is selected if no
    previous candidate has been selected
  \item Select a candidate using the computed probabilities
\end{itemize}

The \textbf{rank-space method} for fitness computation is:
\begin{itemize}
  \item Sort the $n$ individuals by quality
  \item Sort the $n$ individuals by the sum of their inverse
    squared distances to already selected candidates (the lower the
    sum, the better the rank)
  \item Use the rank method, but sort on the sum of the quality
    rank and the diversity rank, rather than on quality rank only
\end{itemize}

\subsection{Crossover}

\begin{itemize}
  \item Crossover enables to search high-dimensional spaces efficiently; 
it reduces the dimensionality of the optimum search space
  \item Crossover enables genetic algorithms to travers obstructing
    moats
\end{itemize}

\subsection{Natural selection}

To mimic natural selection in general:
\begin{itemize}
  \item Create an initial population of one chromosome
  \item Mutate one or more genes in one or more of the current
    chromosomes, producitng one new offspring for each chromosome
    mutated
  \item Mate one or more pairs of chromosomes
  \item Add the mutated na doffspring chromosomes to the current
    population
  \item Create a new generation by keeping the best of the current
    population's chromosomes, along with other chromosomes selected
    randomly from the current population. Bias the random selection
    according to assessed fitness
\end{itemize}

