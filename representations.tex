\section{Representations: classes, trajectories, transitions}

\paragraph{Frame} A frame is formed a node and the links connected
to it in a semantic net

An \textbf{instance constructor} makes instance frames.

A \textbf{slot writer} installs slot values. Its input is a frame,
then name of the slot, and a value to be installed.

A \textbf{slot reader}  retrieves slot values. Its input is a
frame and the name of a slot; its output is the corresponding slot
value.

The slots in an instance are determined by that instance's
superclass: if a superclass has a slot, then the instance inherits
that slot.

Shared knowledge, located centrally, is:
\begin{itemize}
  \item Easier to construct when you write it down
  \item Easier to correct when you make a mistake
  \item Easier to keep yp to date as times change
  \item Easier to distribute because it can be distributed
    automatically
\end{itemize}

One way to accomplish knowledge sharing is to use
\textbf{whe-constructed procedures} associated with the classes of
which the instance is a member. THe expectations established by
when-constructed procedures are called \textbf{defaults}.

A \textbf{class-precedence list} is an ordered list that flattens
a class inheritance structure.

A procedure that is specialized to one of the classes on the
class-precedence list is said to be applicable.

Each class should appear on class-precedence lists before any of
its superclasses.

Each direct superclass of a given class should appear on
class-precedence lists before any other direct superclass that is
to its right.

TO BE CONTINUED
