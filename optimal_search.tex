\section{Nets and Optimal Search}

\subsection{British Museum procedure}

Find all possible paths and select the best one from them.

\subsection{Branch and Bound}

\begin{itemize}
  \item Form a one-element queue consisting of a zero-length path 
    that contains only the root node
  \item Until the first path in the queue terminates at the 
    goal node or the queue is empth:
    \begin{itemize}
      \item Remove the first path from the queue; create new paths 
        by extending the first path to all the neighbors
        of the terminal node
      \item Reject all new paths with loops
      \item add the remaining new paths, if any, to the queue
      \item Sort the entire queue by path length with 
        least-cost paths in front
    \end{itemize}
  \item If the goal node is gound, annouce success; otherwise, 
    annouce failure
\end{itemize}

\subsection{Branch and Bound with lower-bound estimate}

\begin{itemize}
  \item Form a one-element queue consisting of a zero-length path 
    that contains only the root node
  \item Until the first path in the queue terminates at the 
    goal node or the queue is empth:
    \begin{itemize}
      \item Remove the first path from the queue; create new paths 
        by extending the first path to all the neighbors
        of the terminal node
      \item Reject all new paths with loops
      \item Add the remaining new paths, if any, to the queue
      \item Sort the entire queue by \textbf{the sum of the path
        length and a lower-bound estimate of the cost
        remaining, with least-cost paths in front}
    \end{itemize}
  \item If the goal node is gound, annouce success; otherwise, 
    annouce failure
\end{itemize}

\subsection{Branch and Bound with dynamic programming}

\begin{itemize}
  \item Form a one-element queue consisting of a zero-length path 
    that contains only the root node
  \item Until the first path in the queue terminates at the 
    goal node or the queue is empth:
    \begin{itemize}
      \item Remove the first path from the queue; create new paths 
        by extending the first path to all the neighbors
        of the terminal node
      \item Reject all new paths with loops
      \item Add the remaining new paths, if any, to the queue
      \item \textbf{if two or more paths reach a common node,
        delete all those paths except the one that reaches the
        common node with the minimum cost}
      \item Sort the entire queue by \textbf{the sum of the path
        length with least-cost paths in front}
    \end{itemize}
  \item If the goal node is gound, annouce success; otherwise, 
    annouce failure
\end{itemize}

\subsection{A* procedure - Branch and bound with Underestimates
and Dynamic Programming}

\begin{itemize}
  \item Form a one-element queue consisting of a zero-length path 
    that contains only the root node
  \item Until the first path in the queue terminates at the 
    goal node or the queue is empth:
    \begin{itemize}
      \item Remove the first path from the queue; create new paths 
        by extending the first path to all the neighbors
        of the terminal node
      \item Reject all new paths with loops
      \item Add the remaining new paths, if any, to the queue
      \item \textbf{if two or more paths reach a common node,
        delete all those paths except the one that reaches the
        common node with the minimum cost}
      \item Sort the entire queue by \textbf{the sum of the path
        length and a lower-bound estimate of the cost
        remaining, with least-cost paths in front}
    \end{itemize}
  \item If the goal node is gound, annouce success; otherwise, 
    annouce failure
\end{itemize}

\subsection{Which optimal search method is good for me?}
\begin{itemize}
  \item The British Museum procedure is good only when the 
    search tree is small
  \item Branch-and-bound search is good when the tree is big 
    and bad paths turn distinctly bad quickly
  \item Branch-and-bound search with a guess is good when 
    there is a good lower-bound estimate of the distance
    remaining to the goal
  \item Dynamic programming is good when many paths convers
    on the same place
  \item The A* procedure is good when both branch-and-bound 
    search with a guess and dynamic programming are good
\end{itemize}

\section{Trees and Adversarial Search}

A \textbf{game tree} is a representation:
\begin{itemize}
  \item Nodes denote board configuration
  \item Branches denote moves
\end{itemize}
With writers that:
\begin{itemize}
  \item Establish that a node is for the maximizer or for
    the minimizer
  \item Connect a board configuration with a board-configuration
    description
\end{itemize}
With readers that:
\begin{itemize}
  \item Determine whether the node if for the minimizer
    or for the maximizer
  \item Produce a board configuration's description
\end{itemize}

\subsection{Min-Max procedure}

\begin{itemize}
  \item If the limit of the search has been reached, compute 
    the static value of the current position relative to the
    appropriate player. Report the result.
  \item Otherwise, if the level is a minimizing level, use
    MINMAX on the children of the current position. Report 
    the minimum of the results
  \item Otherwise, the level is a minimizing level. Use
    MINMAX on the children of the current position. Report 
    the maximum of the results
\end{itemize}

\subsection{Alpha-Beta procedure}

\begin{itemize}
  \item If the level is the top level, let alpha be
    \(-\infty\) and let beta be \(+\infty\)
  \item If the limit of search has been reached, compute the
    static value of the current position relative to the
    appropriate player. Report the result
  \item If the level is a minimizing level:
    \begin{itemize}
      \item Until all children are examined with ALPHA-BETA or
        until alpha is equal to or greater than beta:
        \begin{itemize}
          \item Use ALPHA-BETA procedure, with the current alpha
            and beta values, on a child; note the value reported
          \item Compare the value reported with the beta value;
            if the reported value is smaller, reset beta to the
            new value
          \item Report beta
        \end{itemize}
    \end{itemize}
      \item Otherwise, the level is a maximizing level:
    \begin{itemize}
      \item Until all children are examined with ALPHA-BETA or
        until alpha is equal to or greater than beta:
        \begin{itemize}
          \item Use ALPHA-BETA procedure, with the current alpha
            and beta values, on a child; note the value reported
          \item Compare the value reported with the beta value;
            if the reported value is larger, reset alpha to the
            new value
          \item Report alpha
        \end{itemize}
    \end{itemize}
\end{itemize}

\subsection{Heuristic Methods}

\subsubsection{Progressive deepening}

For each depth level, compute the best move. By doing so, to compute up to level $d-1$ it is necessary to
evaluate:

\begin{math}
  b^0 + b^1 + \cdots + b^{d-1} = \frac{b^d-1}{b-1} \approx b-1
\end{math}

\subsubsection{Forced move heuristic}

The child move involving a \textit{forced move} generally has a
value that stands out from the rest.

\subsubsection{Singular-extension heuristic}

The search should continue as long as one move's static value
stands out.

\subsubsection{Tapered search}

Arrange for the branching factor to vary with depth of
penetration, possibly using tapered search to direct more effort
into the more promising moves.

